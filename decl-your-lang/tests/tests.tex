\chapter{Testing with SPT}
\chapterlabel{tests}

Testing is indispensible during language development to validate that your
language definition corresponds to your mental design of the language. Automatic
testing is also indispensible for maintenance of your language; does your
language extension still support all the original use cases?

In this chapter we discuss how to test a Spoofax language definition using the
SPT testing language \cite{KatsVV11,KatsVV11a} and using plain example files.
You can find the code for this chapter in example project \LanguageRepoRef{LanguageA}.
The project defines a language with the name \texttt{LangA}.

\section{Testing with SPT}

We start by adding a \texttt{test/} directory to our project and in that
directory we create \texttt{test/expr-syntax.spt} to contain tests for the
expressions in our language.

An SPT test module first declares its name and the language that its tests are
about:

\begin{lstlisting}[language=SPT]
module expr-syntax
language LangA
\end{lstlisting}

Next it declares the start symbol with respect to which the tests will be
parsed:

\begin{lstlisting}[language=SPT]
start symbol Expr
\end{lstlisting}

For each kind of program fragment that should be tested separately, such as
expressions here, the corresponding non-terminal symbol should be declared as
start symbol in the SPT file \emph{and} in the syntax definition, as we will see
in the next chapter.

After these initial declarations follows a series of tests of the form

\begin{lstlisting}[language=SPT]
test <name> [[ <fragment> ]] <property>
\end{lstlisting}

where \texttt{<name>} is the descriptive name of the test (an arbitrary string),
\texttt{<fragment>} is the language fragment to test, and \texttt{<property>}
defines the property that should be tested.
For example, the test

\begin{lstlisting}[language=SPT]
test addition [[
  1 + 2
]] parse succeeds
\end{lstlisting}

declares a test with name \texttt{addition} that states that the string
\texttt{1 + 2} should be parsed succesfully.
 
Tests can also be negative, i.e. require that some operation fails. For example, 
the test

\begin{lstlisting}[language=SPT]
test double plus [[
  1 ++ 2
]] parse fails
\end{lstlisting}

states that the string \texttt{1 ++ 2} should \emph{not} be parsed successfully.
That is, the test would \emph{fail} if the string would be parsed successfully.

In addition to testing for parse success or failure, SPT tests can require a
range of other properties. We will see some of these in further chapters.

\section{Running Tests}

SPT tests are evaluated automatically when the test module is open in Eclipse.
Thus, after modifying and building a language definition, all tests in open test
modules are updated immediately.
The editor identifies using error markers the tests that fail as illustrated in 
\Figure{expr-syntax.spt}.

An alternative way of running tests is by invoking the Spoofax Test Runner (via
the Transform menu), which invokes all test in the current test suite or all
test suites in the project. As illustrated in \Figure{spoofax-test-runner}, the
test runner then shows a list of all tests colored green or red to indicate that
they passed or failed.

\begin{figure}[t]
\includegraphics[width=\hsize]{tests/expr-syntax-spt.pdf}
\caption{Live evaluation of SPT test module with failing test case.}
\figurelabel{expr-syntax.spt}
\end{figure}

\begin{figure}[t]
\includegraphics[width=\hsize]{tests/spoofax-test-runner.pdf}
\caption{Running SPT tests using the Spoofax Test Runner invoked from the
Transform menu.}
\figurelabel{spoofax-test-runner}
\end{figure}

\Figure{class.spt} illustrates another feature of SPT. The code fragment
in a test is handled as text in the language under test, separate from the 
testing language itself. 
For example, the syntax highlighting follows the syntax of the program under
test, keywords of the testing language are not keywords of the language under
test, and syntax errors according to the syntax under test are hightlighted.

\begin{figure}[t]
\includegraphics[width=\hsize]{tests/class-spt.pdf}
\caption{Language-specific syntax highlighting and syntax checks in test
templates.}
\figurelabel{class.spt}
\end{figure}


\section{Examples}

In addition to SPT tests, I typically maintain an \texttt{examples/} directory
with example programs in the language under development. Such examples are
useful to experience the IDE under development and exercise the various
operations being defined on programs.

\section{Further Reading}

The OOPSLA/SPLASH 2011 papers \cite{KatsVV11,KatsVV11a} provide a complete
description of the design and implementation of SPT.
The \href{http://metaborg.org/spt/}{SPT} page on the metaborg site provides a
complete overview of the testing features currrently supported by the
language.


