\chapter{Transformation with Stratego}
\chapterlabel{transformation}

The parsers generated from SDF3 syntax definitions produce abstract syntax
terms. Semantic manipulation of programs in Spoofax is done by manipulating
these terms. In this chapter we study the definition of transformations on
(abstract syntax) terms with the Stratego strategic rewriting language. The code
in this chapter can be found in the \LanguageRepoRef{LanguageB} project.

\begin{figure}[p]
\begin{minipage}[t]{0.47\hsize}
\lstinputlisting[language=paplj]{../languages/LanguageB/examples/expr/let-do.lb}
\end{minipage}
\hfill
\begin{minipage}[t]{0.47\hsize}
\lstinputlisting[language=paplj]{../languages/LanguageB/examples/expr/let-do.desugared.lb.txt}
\end{minipage}
\caption{Program before and after desugaring (concrete syntax).}
\bigskip
\figurelabel{desugaring-concrete}
\lstinputlisting[language=aterm]{../languages/LanguageB/examples/expr/let-do.aterm.txt}
\lstinputlisting[language=aterm]{../languages/LanguageB/examples/expr/let-do.desugared.aterm.txt}
\caption{Program before and after desugaring (abstract syntax).}
\figurelabel{desugaring-abstract}
\end{figure}

\section{Term Transformations}

A desugaring transformation simplifies the programs of a language to use only a
subset of the constructs of the language or only simplified applications of
constructs. 
For example, \Figure{desugaring-concrete} shows a program before and after
desugaring let bindings and sequential compositions. After desugaring, let
bindings contain only a single binding sequential composition `blocks' contain
at most two expresssions (and no `skips').
\Figure{desugaring-abstract} shows the abstract syntax term of the same
program before and after desugaring.

\Figure{Terms.sdf3} defines the syntax of the format of terms that we use here.
A term is a string literal, a number literal, a constructor application, or a
list of terms (between square brackets.)
The algebraic signature corresponding to a syntax definition describes the
subset of terms that are well-formed abstract syntax terms.

Thus, defining a transformation such as desugaring on programs, requires a
transformation on the underlying abstract syntax terms. 
The Stratego language supports the definition of such transformations by means
of term rewrite rules.

\begin{figure}
\lstinputlisting[language=SDF]{../languages/Stratego/syntax/Terms.sdf3}
\caption{Syntax of terms.}
\figurelabel{Terms.sdf3}
\end{figure}


\section{Term Rewrite Rules}

A term rewrite rule is a schematic description of a transformation on terms.
For example, the rule

\begin{lstlisting}[language=Stratego]
desugar-let :
  Let([b1, b2 | bs], e) -> Let([b1], Let([b2 | bs], e))
\end{lstlisting}

transforms a \texttt{Let} term with a list containing at least two bindings
(\texttt{b1} and \texttt{b2}) and some other bindings (\texttt{bs}), into a
\texttt{Let} term with the first binding \texttt{b1} and a nested \texttt{Let}
with the other bindings.
In general, a term rewrite rule has the form

\begin{lstlisting}[language=Stratego]
l : lhs -> rhs
\end{lstlisting}

where \texttt{l} is the name of the rule, and \texttt{lhs} and \texttt{rhs} are
\emph{term patterns}.
A rewrite rule applies to a term if the left-hand side pattern \emph{matches}
the term, and results in the instantiation of the right-hand side pattern,
replacing the pattern variables with the terms bound to them by the match.

\Figure{desugar-rules.str} gives more examples of desugaring rules on
expressions.


\begin{figure}[t]
\lstinputlisting[language=Stratego]{../languages/LanguageB/trans/desugar-rules.str}
\caption{Term rewrite rules for desugaring expressions.}
\figurelabel{desugar-rules.str}
\end{figure}


\paragraph{Term Patterns and Pattern Matching}

A term \texttt{pattern} is a term extended with variables and list match
patterns as formalized by the syntax definition in \Figure{Patterns.sdf3}.
Roughly, a term matches a pattern if it has the same
shape as the pattern. That is, a term matches a pattern if its outermost constructor is the
same as the outermost constructor of the pattern, and the sub-terms match the
corresponding sub-patterns, or the pattern is a variable.
\Figure{pattern-matching} gives a precise definition of pattern matching.

\begin{figure}[p]
\lstinputlisting[language=SDF]{../languages/Stratego/syntax/Patterns.sdf3}
\caption{Syntax of patterns.}
\figurelabel{Patterns.sdf3}
\bigbreak
\begin{boxedminipage}{\hsize}
A term matches a term pattern if one of the following cases aplies
\begin{itemize}
  \item The pattern is a string literal and the term is the same string literal.
  \item The pattern is a number literal and the term is the same number literal.
  \item The pattern is a constructor application \oldtexttt{c(p$_1$,..,p$_n$)}, the
  term is a constructor application \oldtexttt{c(t$_1$,..,t$_n$)} of the same
  constructor \oldtexttt{c}, and the sub-terms
  \oldtexttt{t$_1$,..,t$_n$} match the sub-patterns \oldtexttt{p$_1$,..,p$_n$}.
  \item The pattern is a list pattern \oldtexttt{[p$_1$,..,p$_n$]}, the term is a list
  \oldtexttt{[t$_1$,..,t$_n$]}, and the terms \oldtexttt{t$1$,..,t$_n$} match
  the patterns \oldtexttt{p$_1$,..,p$_n$} .
  \item The pattern is a list pattern \oldtexttt{[p$_1$,..,p$_n$|p]}, the term is a list
  \oldtexttt{[t$_1$,..,t$_m$]}, $n \leq m$, the terms
  \oldtexttt{t$_1$,..,t$_n$} match the patterns \oldtexttt{p$_1$,..,p$_n$},
  and \oldtexttt{[t$_{m+1}$,..,t$_m$]} matches the pattern \oldtexttt{p}.
  \item The pattern is a pattern variable, in which case the term is bound to
  the variable.
\end{itemize}
\end{boxedminipage}
\caption{Definition of term pattern matching.}
\figurelabel{pattern-matching}
\end{figure}


\section{Term Rewriting Strategies}

Term rewriting is the process of applying term rewrite rules to (the sub-terms
of) a term until a \emph{normal form} is reached. That is, until no more
sub-terms match the left-hand side of a rule. In traditional rewrite engines,
rules are applied using a standard \emph{strategy} to find \emph{reducible
expressions} (or \emph{redices}).
However, in practice rewrite rules (for program transformation) may be
\emph{non-confluent}, i.e. produce multiple different results depending on the
application order, and/or may be \emph{non-terminating}. Circumventing such
problems in a traditional rewrite engine requires extending terms with
additional information in order to control execution.

In order to provide more control over the rewriting process, Stratego supports
\emph{programmable} rewriting strategies, which make it possible to
\emph{select} the rules to apply and to determine the search algorithm used to
find redices.
Indeed, such strategies do not necessarily lead to a (unique) normal form in the
traditional sense.

The Stratego library provides a large collection of generic (traversal)
strategies. These strategies are generic in the sense that they do not depend on
the abstract syntax of a particular language. For example, \Figure{desugar.str}
uses the \texttt{innermost} strategy from the library to define several
desugaring transformations using the desugaring rules from
\Figure{desugar-rules.str}.

\begin{figure}[t]
\lstinputlisting[language=Stratego]{../languages/LanguageB/trans/desugar.str}
\caption{Strategies for desugaring expressions.}
\figurelabel{desugar.str}
\end{figure}

\section{Defining Strategies}

\begin{figure}[t]
\lstinputlisting[language=Stratego]{../languages/LanguageB/trans/stratego-lib/traversals.str}
\caption{Definitions of traversal strategies (included in the standard
Stratego library).}
\figurelabel{traversals.str}
\end{figure}

A Stratego strategy is a partial function from terms to terms. A strategy is a
partial function since the application to a term may fail. For example, applying
a rewrite rule to a term fails if the term does not match the left-hand side of
the rule. Strategies are defined in terms of a small set of basic
transformation operations:

\begin{description}
\item[\oldtexttt{id}] The identity strategy always succeeds and returns the term
to which it is applied.
\item[\oldtexttt{fail}] The failure strategy always fails.
\item[\oldtexttt{f}] The application of a rewrite rule or strategy
\oldtexttt{f} to the current term. If successful, returns the instantiation of
the right-hand side of the rule.
\item[\oldtexttt{<f>p}] The application of a rewrite rule or strategy
\oldtexttt{f} to the term pattern \oldtexttt{p}.
\item[\oldtexttt{?p}] Match the term against the pattern \oldtexttt{p}.
\item[\oldtexttt{!p}] Replace the term by instantiating the pattern
\oldtexttt{p} (also known as \emph{build}).
\end{description}

These basic transformation operations are combined using strategy
\emph{combinators}:

\begin{description}
\item[\oldtexttt{s$_1$; s$_2$}] The sequential composition of two strategies.
\oldtexttt{s$_2$} is applied to the term returned by the
application of \oldtexttt{s$_1$}. Fails if either strategy fails.
\item[\oldtexttt{s$_1$ <+ s$_2$}] The ordered choice between two strategies.
If the application of \oldtexttt{s$_1$} is successful, its result is returned,
otherwise \oldtexttt{s$_2$} is applied to the original term.
\item[\oldtexttt{all(s)}] A one-level traversal operator that visit all direct
sub-terms of a term, applying \oldtexttt{s} to each, returning the application
of the original constructor to the transformed sub-terms.
\item[\oldtexttt{one(s)}] This one-level traversal operator visits the direct
sub-terms of a term from left to right, applying \oldtexttt{s} until it
succeeds, returning the original term with the first sub-term for which
\oldtexttt{s} succeeds replaced.
\item[\oldtexttt{crush(n,c,s)}] Apply strategy \oldtexttt{s} to the direct
sub-terms of a term and fold the resulting list using the nil and cons
operations \oldtexttt{n} and \oldtexttt{c}.
\end{description}

\Figure{traversals.str} shows the application of these combinators in the
definition of several strategies from the standard Stratego library.

\section{Parameterized and Conditional Rules}

In \Figure{traversals.str} we saw that strategy definitions can be parameterized
with strategy arguments. The strategy \strcode{bottomup(s)} takes a strategy
parameter \texttt{s} that is instantiated with some transformation to apply to
each node of a term. Rules can be parameterized in the same way.
\Figure{list-strategies.str} gives examples of such parameterized rules. For
example, the \texttt{map(s)} rule applies a transformation \strcode{s} to all the
elements of a list, and \strcode{filter(s)} filters the elements of a list using
(the success/failure behaviour of) the transformation parameter \strcode{s}. 
Note that \texttt{<s>x} denotes the application of the strategy (parameter)
\strcode{s} to term (variable) \strcode{x}.

Rules and strategies have two parameter lists, one for by-name parameters, and
one for by-value parameters, which are separated by a bar \texttt{|}. Thus, in
general a strategy call has the form
\oldtexttt{f(s$_1$,..,s$_n$|t$_1$,..,t$_m$)}, where the \oldtexttt{s$_i$} are
strategy (by-name) parameters, and the \oldtexttt{t$_i$} are term (by-value)
parameters. The bar can be left out if there are no by-value parameters.
For example, in the \texttt{inc(|n)} strategy takes a term parameter \texttt{n}
and calls the \texttt{map} strategy with a strategy parameter
\texttt{\\ x -> <add>(x, n) \\}, which will be applied to each element of the
list to which \texttt{inc(|n)} is applied.

Rules can have side conditions prefixed with \strcode{where} or \strcode{with}
followed by a strategy expression. Both kinds can compute an intermediate result
needed by the next condition or the right-hand side of the rule. (A pattern
matching assignment \strcode{p := t} matches the result of the \strcode{t}
against the pattern \strcode{p}.) In addition, a \strcode{where s} condition
tests whether its strategy argument succeeds or fails; in the latter case, the
rule fails. For example, the
\strcode{collect-all(s)} strategy in \Figure{list-strategies.str} uses the
condition \strcode{where t' := <s>t} to test whether the application
\strcode{<s>t} succeeds; if it does not, the next rule is tried. On the other
hand, a \strcode{with s} condition expects its strategy argument \strcode{s} to
succeed; if it does not, the program throws an exception. For example, the
\strcode{collect-all(s)} strategy uses the condition \strcode{with ts :=
<crush(![], union, collect-all(s))>t} to compute the collection from the
sub-terms of \strcode{t} and expects this to always succeed.

\begin{figure}[t]
\lstinputlisting[language=Stratego]{../languages/LanguageB/trans/stratego-lib/list-strategies.str}
\caption{Parameterized rules.}
\figurelabel{list-strategies.str}
\end{figure}

\section{String Interpolation}

The proper way to define a translation from one language to another is through
'code generation by model transformation'. That is, transform
the abstract syntax terms from the source language to abstract syntax terms of
the target language and then pretty-print the resulting term. This separates
program composition from pretty-printing and allows transformations to be
applied to the generated target code.
Since I'm in a hurry to get to the next chapters, I will only discuss a
quick-and-dirty approach to code generation using \emph{string interpolation}
(for now), and refer to the further reading section for references to code
generation by model transformation.

Code generation by string composition has a bad reputation. Among its many
short-comings, the most mundane one is quotation hell. The characters of a
string are enclosed in double quotes. Any double quotes to be included in the
string need to be escaped. When the language does not support multi-line
strings, newlines have to be included as escaped characters.
Splicing in strings resulting from a sub-computation, such as a recursive
invocation of the translation function, requires breaking the string into
sub-strings and composing those with a string concatenation operator. The result
is typically quite unreadable.

String interpolation (or string templates) addresses these problems (and is
supported by more and more programming languages). In Stratego a string
interpolation expression has the form \strcode{$[..[..]..]}, where the
(multi-line) text between \strcode{$[} and \strcode{]} is taken as literal
characters, while the embedded \strcode{[..]} expressions are escapes to splice
in the values of variables or the results of sub-computations. (To avoid
quotation conflicts one can choose different quotation symbols such as
\strcode{$<..<>..>}.) For example, \Figure{to-xml.str} defines the translation
of programs in our example language to XML documents. The body of the string
template for the \strcode{Program} rule consists of the literal XML pattern for
a \strcode{<program>} document. Using escapes, the translations of the sub-terms
(\strcode{name}, \strcode{cs}, and \strcode{e}) are inserted into the template.
Note the difference between the literal \strcode{name} attribute and the
splicing of the \strcode{name} variable in \strcode{name="[name]"}, and the
application of strategy expressions in escapes such as as
\strcode{[<map(to-xml)> cs]}.

In addition to solving quotation hell, string templates take care of getting
indentation right. The common prefix (caused by the indentation of the
meta-program) is discarded. On the other hand, relevant indentation, such
as the indentation of classes in \Figure{to-xml.str}, is applied uniformly to
the lines in the text spliced into the template.

\begin{figure}[t]
\lstinputlisting[language=Stratego]{../languages/LanguageB/trans/to-xml/to-xml.str}
\caption{Using string interpolation to translate programs to XML documents.}
\figurelabel{to-xml.str}
\end{figure}


\section{Debugging}

Unfortunately, Stratego lacks an interactive debugger that let's you step
through the execution of your code. So, the usual approach to debugging Stratego
programs is pretty classical. First, reduce the input to a transformation to the
smallest term that exhibits the unexpected behavior. Second, use good old
`println' debugging to inspect to where a program proceeds and inspect run-time
values. The stratego library defines the \strcode{dbg(|msg)} strategy, which
prints the current term prefixed with \strcode{msg}.

\section{Spoofax Builders}

We have defined two transformations on the abstract syntax trees of our example
language, desugaring and generation of XML documents.
Now we would like to apply these transformations in the editor for our language. 
This requires the definition of \emph{builders}. 
Builders are actions that can be applied to a program from the editor. 
The definition of these actions consists of three ingredients. 
We already took care of the first ingredient, the definition of the
transformation to apply.

Second, we need to define menu entries that let us invoke the transformations.
To that end, we create a new editor services file
\texttt{LangB-Menu-Transformation.esv} in the \texttt{editor/} directory and
import it in the main \texttt{LangB-Menus.esv} module.
In the module, we declare a menu with the name \texttt{Transformation} and
containing three `actions' as show in \Figure{LangB-Menu-Transformation.esv}.
These actions associate a menu entry with a name (e.g. \texttt{Desugar}) and a
strategy to apply (e.g. \texttt{editor-desugar}).

\begin{figure}[t]
\lstinputlisting[language=ESV]{../languages/LanguageB/editor/LangB-Menu-Transformation.esv}
\caption{Configuration of Transformation menu.}
\figurelabel{LangB-Menu-Transformation.esv}
\end{figure}

\begin{figure}[t]
\lstinputlisting[language=Stratego]{../languages/LanguageB/trans/builders/transformation.str}
\caption{Implementation of builders for the Transformation menu.}
\figurelabel{builders/transformation.str}
\end{figure}

Third, we need to define the glue code that implements the interface between a
menu and the transformation strategy. \Figure{builders/transformation.str}
defines the strategies that implement the builders for the three menu items in
\Figure{LangB-Menu-Transformation.esv}. Each of these define a transformation
from a tuple \strcode{(selected, position, ast, path, project-path)} ---
consisting of the selected AST, the position of the cursor, the entire AST of
the editor, the path name of the file, and the path name of the project
directory --- to a pair \strcode{(filename, result)}, consisting of a file name
to write the result to and the string representation of the result.

Useful strategies from the Stratego library to use here include
\strcode{get-extension} to get the file extension from a path, and
\strcode{guarantee-extension(|ext)} to replace the extension of a file name with
\texttt{ext}.

%\section{Exercises}

\section{Further Reading}

Stratego emerged from my experience with term rewriting in the ASF+SDF
MetaEnvironment \cite{Klint93,BrandDHJJKKMOSVVV01}. The ASF formalism uses
equations to define algebraic equivalences of the terms induced by an algebraic
signature. It was implemented using a rewriting engine based on innermost
exhaustive reduction. The need to control the application of rules, prompted the
introduction of auxiliary function symbols. 

Inspired by the notion of strategies in ELAN
\cite{BorovanskyKKMV96,BorovanskyKKMR98} and the modal operators of modal logic,
Luttik and I introduced one-level traversal combinators (all, some, one) for
generic term traversal \cite{LV97}.
These provided the basis for the design of the Stratego transformation language
\cite{VisserBT98} and System S as a core language for rewriting
\cite{VisserB98}. The Stratego only appeared later in publication
\cite{Visser01}, although it was used from 1998.

Stratego was first implemented as an interpreter in ML. Using that interpreter a
compiler from Stratego to C was defined in Stratego and then bootstrapped. 
The compiler is organized in a front-end that reduces Stratego Sugar to Stratego
Core (which may be stored in .ctree files), and a back-end that translates
to a target language. After multiple iterations of the C translation scheme, the
back-end of the compiler was ported to produce Java code around 2006, which is
used by Spoofax. In addition to a compiler, Stratego is supported by an
interpreter implemented in Java.

Programming in the Stratego language is supported by a set of libraries that
provide standard transformation utilities, such as parsing and pretty-printing.
Originally these libraries were developed under the name of the XT
transformation tools. Several iterations of these tools along with the
development of the Stratego are described in publications: XT \cite{JongeVV01},
Stratego/XT 0.9 \cite{Visser03}, Stratego/XT 0.17 \cite{BravenboerKVV08}. 
After the 0.17 release of Stratego/XT, the language was ported to Java and
Stratego/XT is now distributed as a JAR, including the compiler and several
pre-compiled Stratego libraries.
The Stratego/XT distribution is now included in the distribution of the Spoofax
Language Workbench \cite{KatsV10}.

Stratego supports meta-programming with concrete object syntax, which entails
that terms in rewrite rules can be written using the concrete syntax of the
object language \cite{Visser02}. 
This approach is a key ingredient of the `code generation by model
transformation' approach \cite{HemelKGV10}.

%references to string interpolation

\paragraph{Online Documentation}

The online
\href{http://hydra.nixos.org/job/strategoxt-docs/strategoxt-manual/html/latest/download/1/manual/chunk-chapter/index.html}{StrategoXT
Manual} provides a fairly complete overview of
the Stratego language and the XT toolset \cite{StrategoXTReferenceManual}.
While the tool aspects are not up-to-date with the current Java-based libraries,
the \href{http://hydra.nixos.org/job/strategoxt-docs/strategoxt-manual/html/latest/download/1/manual/chunk-chapter/tutorial.html}{Stratego/XT Tutorial} 
provides an extended introduction to the language.


%\href{http://releases.strategoxt.org}{releases.strategoxt.org}

